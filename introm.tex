\chapter{Introduction}
A variety of IoT (Internet-of-Things)
systems
are already widely available on the market.
These systems are typically controlled by \textit{event-driven} smart apps
that take as input either sensed data, user inputs, or other external triggers (from the Internet)
and command one or more actuators towards providing different forms of automation.
Examples of sensors include smoke detectors, motion sensors, and contact sensors.
Examples of actuators include smart locks, smart power {\color{black}outlets}, and door controls.
Popular control platforms on which third-party developers can build smart apps
that interact wirelessly with these sensors and actuators include
Samsung's SmartThings~\cite{Samsung:smartthings}, Apple's HomeKit~\cite{Apple:homekit},
and
%Google's Weave/Brillo~\cite{Google:weave}, among others.
Amazon's Alexa~\cite{Amazon:alexa}, among others.
%Vera's Smart Home Controller~\cite{Vera:homecontroller},
%Intel's Smart Buildings~\cite{Intel:smartbuildings}, {\color{violet}Logitech Harmony~\cite{Logitech:harmony}, and Microsoft Azure IoT~\cite{Microsoft:iot}.}

While conceivably, IoT is here to stay,
current research studies on security/safety of IoT systems are limited in two fronts \cite{2018arXiv181009551N}.
First, they focus on \emph{individual components} of IoT systems:
there are papers on the security of communication protocols
\cite{Dolly2016,Fouladi2013,Ho2016:smartlock,Lomas:zigbeeflaw,Eyal:iotworm,doi:10.1177/1550147718767605,5476636,5506358},
firmware of devices
\cite{7815045,6983801,8047972,7483485,7581459,Costin:analysis},
{\color{black}platforms}
\cite{Earlence:flowfence,Jia:contexiot},
and smart apps
\cite{Earlence:smarthomesecurityanalysis,Earlence:flowfence,Jia:contexiot,203866}.
%Only
{\color{black}Very few efforts have} %\zhiyun{in abstract we say no work??} 
taken a holistic perspective of \emph{an IoT system}.
Second, most current research efforts only focus on securing the cyberspace,
and do not address the safety and security of the physical space,
which is one of the key obstacles for real-world IoT deployment~\cite{iot_security_news,2018arXiv180906962B}.
%Third, they do not consider device failure \cite{Samsung:smartthingscom1,Samsung:smartthingscom2, Samsung:smartthingscom3, Samsung:smartthingscom4}, which may cause an IoT system into bad states.}

Our thesis is that a holistic view of an IoT system is important \ie,
the distributed sensors and actuators, and the apps
that interact with them need to be considered jointly.
While the compromise of an individual component may lead to
the compromise of the whole system,
certain complex security and safety issues are only revealed when the interactions
between components (\eg, a plurality of poorly designed apps) and/or possible {\color{black}device/communication failures}
are considered.
%For example, badly designed apps (not necessarily malicious),
%or undesirable (and in many cases unforeseen) interactions between apps installed by
%a user can drive the IoT system into dangerous states.
These latent problems are very real since apps are often developed by
third-party vendors without coordination,
and are likely to be installed by one or more users (\eg, family members) at different times.
%(or by multiple users such as members of a family);
%while users are unlikely to realize that apps may interact in bad ways.
{\color{black}
Moreover, both legitimate device failures~\cite{Samsung:smartthingscom1,Samsung:smartthingscom2, Samsung:smartthingscom3, Samsung:smartthingscom4} (\eg, from battery depletion) and induced
communication failures
(\eg, via jamming~\cite{5473884})
%failures %or
%communication failures (\eg, jammed by attackers) 
can lead to missed interactions between autonomous components, which in turn can
cause
the entire system to transition into a bad state. 
}
%due to loss in interactions between
%autonomous components.
These issues are especially dangerous,
because bad or missed interactions can be deliberately induced by attackers
via {\color{black} spoofing sensors  
\cite{son2015rocking,shin2016sampling},
luring users to install malicious apps~\cite{Jia:contexiot},
or jamming sensor reports}.
%The above issues are especially dangerous because
%adversaries can potentially exploit
%inherent logical flaws arising due to these interactions,
%to cause and perhaps exploit unsafe physical states.

\section{Goals}
%\textbf{Goals}:
In this paper, our goal is to build a holistic \textcolor{black}{system}
%\thomas{Ed's comment: method?}
which,
given an IoT system and
a set of default plus user-defined safety properties with regards to
both the cyber and physical {\color{black}spaces},
(a) finds if components of an IoT system
or interactions between components can lead to bad states that violate these properties;
and, (b) attributes the detected violations to either benign misconfigurations
or potential malicious apps.
With regards to (a) we account for cases wherein
app interactions or 
{\color{black}failed device(s)/communications} can cause a bad state.
With regards to (b) we look for repeated instantiations of unsafe states
since malicious apps are likely to
consistently try to coerce the IoT system into exploitable bad states
(\eg, those described in~\cite{Jia:contexiot}).

To achieve our goal, we need to solve a set of technical challenges.
Among these, the key challenge lies in the scope of the analysis:
as the number of IoT devices and apps is already
large and is only likely to grow in the future~\cite{alpha,iotexp},
{\color{black} physical replication and testing of IoT systems is hard (due
to scale).} 
%and 
%does not scale.}
%it is unscalable to physically replicate a given IoT system
%and test it}.
Thus, it is desirable to build a realistic model of the system,
which captures the interactions between sensors, apps, and actuators.

\section{Our Solution}
%\textbf{Our solution}:
We achieve our goal by addressing the above and 
other practical challenges, in a novel framework \sys
(for IoT Sanitizer).
In brief, \sys uses model checking as a basic building block.
Towards alleviating the state space explosion problem associated with
model checking~\cite{Clarke2012},
we design two optimizations within \sys to
(i) only consider apps that interact with each other,
{\color{black}and (ii) eliminate unnecessary interleaving that is unlikely to yield useful assessment of unsafe behaviors.}
We also design an attribution module which flags potentially malicious apps,
and attributes other unsafe states to bad design or misconfiguration.

%To assess the feasibility and effectiveness of our solution,
We develop a prototype of \sys based on the \spin model checker~\cite{Holzmann:spin}
and apply it to the Samsung SmartThings platform.
As one contribution, we design an automated model generator
that translates apps written in Groovy (the programming language of SmartThings apps)
into Promela, the modeling language of \spin.
{\color{black}To evaluate \sys,
we postulate 45 common sense safety properties} and consider 150 smart apps
with 76 configurations. With this setup,
\sys discovered 147 violations of 20 safety properties
due to app interactions (135 violations) and {\color{black}device/communication failures} (12 violations).
In an extreme case, 4 smart apps needed to interact to cause a violation,
which is extremely difficult to spot manually.
%{\color{black}11 violations were found in configurations of 1 app,
%19 in configurations of 2 apps, 8 in configurations of 3 apps,
%10 in configurations of 4 apps, 2 in configurations of 5 apps,
%and 1 in a configuration of 6 apps.}
%a group of 1, 2, 3, 4, 5, and 6 app; and the corresponding violations were 11, 19, 8, 10, 2, and 1, respectively.
We evaluate our attribution module with 9 malicious apps
from~\cite{Jia:contexiot} that are relevant to our problem scope
(\eg, causing bad physical states).
\sys attributes all 9 apps to be potentially malicious.

\section{Contributions}
%\textbf{Contributions}:
A summary of our contributions is as follows:
\begin{itemize}
%\item To the best of our knowledge, we are the first work using model checking to identify app-level threats of IoT systems.
\item We map the problem of {\color{black} detecting potential safety issues of 
an IoT system} into a model checking problem.
  We develop novel pre-processing methods to alleviate the state explosion problem in model checking.
  %We model IoT systems including different types of devices (sensors and actuators), apps and their interactions, as well as configurations.

\item We design
  \sys to detect safety violations in IoT systems and
  develop a prototype that applies to the Samsung SmartThings platform. {\color{black} We provide the source code of \sys for download at https://github.com/dangtunguyen/IoTSan}.
  We develop tools to automatically translate the app source code into Promela.
  We evaluate \sys with 150 smart apps from the SmartThings' market place
  and discover 147 possible safety violations.

\item We propose a method to attribute safety violations to either bad apps or misconfigurations.
  The method attributes 9 known malicious apps with 100\% accuracy.
\end{itemize}

